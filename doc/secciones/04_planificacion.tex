\chapter{Planificación}

\section{Desarrollo ágil}
La realización de este proyecto se ha llevado a cabo con \textbf{desarrollo ágil}. Este se basa en priorizar la entrega de software funcional, plazos de entregas reducidos, colaboración con el cliente y adaptabilidad frente a cambios. Fue propuesto en 2001, en el \textit{Manifiesto Agile}.\cite{agile}

O de una forma más sencilla, esta metodología nos dice:
\begin{enumerate}
    \item ¿Qué tengo que hacer ahora?
    \item ¿Es correcta la solución que he planteado al problema que estoy resolviendo?
\end{enumerate}

\section{Historias de usuario}
Serán aquellas peticiones que guiarán el desarrollo, es decir son las peticiones a satisfacer para tener un desarrollo correcto. Se han creado las siguientes:
\begin{itemize}
    \item \href{https://github.com/marcosrmartin/MTD_Server/issues/16}{[HU-1]} Adaptar el entorno a las implementaciones: como programador, necesito adaptar el entorno para la implementación de MASS #12 y el MTD que utiliza K8s #21. Tener en cuenta que MASS está escrito en Python y el otro tiene soporte de Maven. Entre mis necesidades se encuentran realizar test sobre las implementaciones, gestionar sus dependencias y facilitar el proceso de instalación de ambas. Todo esto me permitirá trabajar más cómodamente, a la vez que mejorar el flujo de trabajo.
    \item \href{https://github.com/marcosrmartin/MTD_Server/issues/12}{[HU-2]} MASS pruebas: como investigador, necesito realizar \textit{benchmarks} para mi investigación sobre MASS (debido a que es de las últimas tecnologías MTD aplicada a servidores web) pero esta no es \textit{opensource} #15.
    \item \href{https://github.com/marcosrmartin/MTD_Server/issues/21}{[HU-3]} Implementar MTD K8s en el repositorio: como investigador, para llevar a cabo #15, necesito realizar pruebas con https://github.com/ptibom/Moving-Target-Defense-with-Kubernetes.
    \item \href{https://github.com/marcosrmartin/MTD_Server/issues/15}{[HU-n]} Comparativa entre MASS y MTD K8s: como investigador, necesito realizar una comparativa entre las implementaciones actuales de MTD a servidores web. Estas son: MASS y https://github.com/ptibom/Moving-Target-Defense-with-Kubernetes. Para así saber cual es la mejor opción para un entorno de producción.
\end{itemize}
A partir de estas historias se creará el camino a seguir para el desarrollo del proyecto.

\section{Seguimiento del desarrollo - Hitos}
``Los milestones son herramientas para comenzar a trabajar y organizar el trabajo con un objetivo claro y concreto en cada fase.
''\cite{iv}. Es decir, son el camino que nos marcamos antes de empezar a trabajar y estos tendrán una serie de requisitos para poder superarse. Los hitos están basados en las peticiones hechas por los usuarios(HUs). Se han llevado a cabo los siguientes hitos:
\begin{itemize}
    \item \href{https://github.com/marcosrmartin/MTD_Server/milestone/1}{[M-0]} Preparar el entorno para la documentación.
    \item \href{https://github.com/marcosrmartin/MTD_Server/milestone/3}{[M-1]} Infraestructura inicial MASS.
    \item \href{https://github.com/marcosrmartin/MTD_Server/milestone/2}{[M-2]} Implementación MASS.
\end{itemize}

% \section{Temporización}
% Hacer gráfica cuando se tenga el total de horas

\section{Costes}
\begin{table}[H]
	\centering
	\begin{tabular}{| l | l | r |}
        \hline
        \textbf{Concepto} & \textbf{Materiales} & \textbf{Precio} \\
        \hline
        Hardware	& Ordenador y periféricos & Amortización* 308.75 €/año\\
        Personal 	& Ingeniero Junior	& 18000-23000 al año \\
        Software 	& Software libre gratuito & 0 € \\
        % Recursos en la nube 	& GitHub plan gratuito & 0 € \\
        \hline
        % Coste total 			& x horas de desarrollo & x-x € \\
        \hline
	\end{tabular}
	\caption{Costes estimados del proyecto.}
\end{table}

% https://twitter.com/isagasti/status/1593923910617251841
*Amortización aplicando el coeficiente máximo de amortización lineal para el grupo ''equipos para procesos de información''\cite{amortizacion}. Se incluye todo el equipo informático como conjunto operativo. Coste de compra total 1235 €, 1100 € ordenador y 135 € periféricos.