\chapter{Implementación}

La implementación del software se ha dividido en hitos. Estos han sido definidos en Github
y cada uno de ellos contiene un grupo de \textit{issues} que se corresponden con las distintas
mejoras que se han ido incorporando al software a lo largo de su desarrollo.

\section{Toma de decisiones}

\subsection{Gestor de dependencias}
Los criterios para elegir el gestor de dependencias, de mayor a menor prioridad, han sido:
\begin{enumerate}
    \item Soporte pyproject.toml.
    \item Creación de entornos virtuales/gestión las dependencias de forma local.
    \item Rendimiento.
\end{enumerate}

Donde se van a evaluar los siguientes gestores de dependencias: \textit{Pipenv, Poetry, Hatch, PDM, Conda, Mamba, Pixi, Rye, Miniconda y Micromamba}

\begin{table}[H]
    \centering
    \begin{adjustbox}{width=\textwidth, totalheight=\textheight, keepaspectratio}
        \begin{tabular}{|c|c|c|c|c|c|c|c|c|c|c|c|c|}
            \hline
            Requisitos & Pipenv & Poetry & Hatch & PDM & Conda & Mamba & Pixi & Rye & Pip & Pip-tools & Miniconda & Micromamba\\
            \hline
            Pyproject.toml & \checkmark & \checkmark & \checkmark & \checkmark & \ding{55} & \ding{55} & \ding{55} & \checkmark & \ding{55} & \ding{55} & \ding{55} & \ding{55} \\
            Entornos virtuales & \checkmark & \checkmark & \checkmark & \checkmark & \checkmark & \checkmark & \checkmark & \checkmark & \ding{55} & \ding{55}  & \checkmark & \checkmark \\
            \hline
        \end{tabular}
    \end{adjustbox}
    \caption{Comparativa entre gestores de dependencias.}
\end{table}

Pip y pip-tools, no gestionan entornos virtuales, el resto de opciones si lo hace. Conda, mamba y sus versiones lite no tienen pyprojec.toml, en cuanto a pixi, este está basado sobre conda, pero genera en el directorio de trabajo su pixi.lock y pixi.toml, el cual no cumple con el estándar de pyproject.toml.

Para evaluar su rendimiento nos basaremos en diferentes benchmarks\cite{pm-benchmark-shootout}. Donde observamos que poetry es el más rápido en realizar la instalación desde un lockfile, creación de este y en añadir un paquete. Además es el segundo en actualizar los paquetes. Sin embargo es el que más tarda en ser instalado. Ya que el gestor de dependencias será instalado una sola vez, vamos a priorizar los otros benchmarks. Por lo que tenemos a poetry como ganador ante PDM y pipenv.

Debido a que no se ha encontrado ningún benchmark sobre Hatch y Rye (este además de que está en un estado experimental), se utilizará poetry como gestor de dependencias.

\subsection{Gestor de tareas}
Los criterios para elegir el gestor de tareas, de mayor a menor prioridad, han sido:
\begin{enumerate}
    \item Curva de aprendizaje.
    \item Ficheros de configuración.
\end{enumerate}

Donde se van a evaluar los siguientes gestores de tareas: \textit{Poethepoet, Pypyr, Invoke, Doit, Pytask}

\begin{table}[H]
    \centering
    \begin{adjustbox}{width=\textwidth, totalheight=\textheight, keepaspectratio}
        \begin{tabular}{|c|c|c|c|c|c|}
        \hline
        Requisitos & Poethepoet & Pypyr & Invoke & Doit & Pytask\\
        \hline
        Curva de aprendizaje & Fácil & Difícil & Medio & Medio & Medio \\
        Ficheros de configuración & pyproject.toml & pipelines/*.yml + .py & tasks.py & dodo.py & task\_*.py \\
        \hline
        \end{tabular}
    \end{adjustbox}
      \caption{Comparativa entre gestores de tareas.}
\end{table}

De las opciones anteriores, poethepoet es el que menos deuda técnica aporta al proyecto, ya que además de permitir funciones en Python como invoke, doit y pytask, permite utilizar shell, comandos y expresiones en Python directamente. Sumando que utiliza el fichero de configuración pyproject.toml, por lo que se evitarían ficheros extra.


\subsection{Test runner}
Los criterios para elegir el \textit{test runner}, de mayor a menor prioridad, han sido:
\begin{enumerate}
    \item Utiliza BDD (Behavior Driven Development).
    \item Estructura de archivos.
    \item Diferentes características.
\end{enumerate}

Donde se van a evaluar las siguientes herramientas: \textit{Pytest, Nose2, Unittest, Green, Behave, Testcontainers, Radish}

\begin{table}[H]
    \centering
    \begin{adjustbox}{width=\textwidth, totalheight=\textheight, keepaspectratio}
        \begin{tabular}{|c|c|c|c|c|c|c|c|}
        \hline
        Requisitos & Pytest & Nose2 & Unittest & Green & Behave & Testcontainers & Radish \\
        \hline
        Utiliza BDD & \checkmark (plugin) & \ding{55} & \ding{55} & \ding{55} & \checkmark & \ding{55} & \checkmark \\
        \hline
        \end{tabular}
    \end{adjustbox}
      \caption{1ª Comparativa entre gestores de tareas.}
\end{table}

Se descartan todas excepto Pytest, Behave y Radish.

La estructura de archivos entre estos apenas varía, todos utilizan \textit{steps}, \textit{features} y un fichero \textit{.ini} (el cual puede ser utilizado para cambiar la estructura).

En este punto los tres son válidos en el proyecto, sin embargo, vamos a comparar algunas características para seleccionar uno:

\begin{itemize}
    \item Paralelismo: Behave no tiene paralelismo nativo (y el plugin que lo soportaba esta deprecated), mientras que Radish lo soporta de forma nativa y Pytest con un plugin (pytest-xdist).
    \item Escenario como precondición: Radish permite que para ejecutarse un escenario se tenga que cumplir otro previamente.
    \item Bucle de escenario: Radish permite ejecutar escenarios en bucle.
    \item Declaración explícita de escenarios: Pytest necesita que en el fichero \textit{steps} se declare explícitamente el escenario.
    \begin{table}[H]
        \centering
        \begin{adjustbox}{width=\textwidth, totalheight=\textheight, keepaspectratio}
            \begin{tabular}{|c|c|c|c|}
            \hline
            Otras características & Pytest & Behave & Radish \\
            \hline
            Dependencias & 11 & 4 & 9 \\
            \hline
            Tamaño & 1.25 MB & 982 KB & 1.89 MB \\
            \hline
            Forks (Github) & 206 & 679 & 48 \\
            \hline
            Stars (Github) & 1.2 K & 3 K & 176 \\
            \hline
            Snyk advisor  & 91 & 71* & 87 \\
            \hline
            \end{tabular}
        \end{adjustbox}
            \caption{2ª Comparativa entre gestores de tareas.}
    \end{table}
\end{itemize}


En esta comparativa se puede ver que Pytest se queda un poco atrás en la facilidad de uso de BDD, mientras que Radish es el que tiene más funcionalidades. Sin embargo estas funcionalidades, como el paralelismo, no se aplicarán a este proyecto, ya que los test modificarán iptables de la máquina donde se ejecute y levantará contenedores, aunque estas tareas podrían ser paralelizables, debido a su complejidad no se realizarán. Otra funcionalidad como los bucles de escenarios no será necesario y los escenarios como precondición podrían ser útiles en otro contexto, ya que este se podría utilizar para comprobar que el docker está corriendo correctamente, pero esto se adaptaría mejor en el \textit{setup} y \textit{teardown} (utilizando menos recursos), por lo que, aunque Radish brinda más funcionalidades, difícilmente serán adaptables a este proyecto, por lo que se va a utilizar Behave el cual cumple con los requisitos necesarios, es el que añade menos dependencias, es el más ligero y cuenta con mayor comunidad (dentro de BDD). Aunque este es marcado en Snyk advisor\cite{snyk} como inactivo, no lo es, ya que este utiliza la última versión en Pypi, la cual salió en 2018, sin embargo el proyecto se ha seguido desarrollando, solo que no cuenta con una nueva versión en Pypi. En Github se puede encontrar su propio versionado y \href{https://behave.readthedocs.io/en/latest/install/#using-the-github-repository}{como descargarlo}.


\subsection{Librería de aserciones}
Los criterios para elegir la librería de aserciones, de mayor a menor prioridad, han sido:

\begin{enumerate}
    \item Por defecto.
    \item Otras características.
\end{enumerate}

Donde se van a evaluar las siguientes librerías: \textit{Assert, Unittest, PyHamcrest, Pytest, Asserpy, Truth, Matchers, Grappa, Verify}

Se descartarán todos excepto Assert y Unittest, los cuales están incluidos por defecto en Python.

Se utilizará Unittest, ya que aunque no se necesite gran variedad de funcionalidades y se pueda conseguir lo mismo con ambas, Unittest permitirá no tener que escribir explícitamente los fallos (lo cual sería más necesario si no se utilizase un BDD) y usar una función autodescriptiva, consiguiendo una mayor legibilidad.