\chapter{Introducción}
Los servidores web son la cara visible de internet, actualmente hay alrededor de 12'1 millones de servidores web activos en el mundo \cite{netcraft-agosto23}. Estos servidores son el objetivo de muchos ataques, ya que son la puerta de entrada a los datos de los usuarios. Por ello, es importante que los servidores web estén protegidos ante ataques, y que en caso de que se produzca un ataque, el servidor sea capaz de recuperarse y seguir ofreciendo servicio.

Hay diferentes métodos para aumentar la seguridad de un servidor web, como por ejemplo, la utilización de cortafuegos, WAFs, protección contra ataques de denegación de servicio, sistemas de detección y prevención de intrusiones, etc \cite{medium-attacks}. Sin embargo, estos métodos tienen algo en común, y es que ninguno cambia el entorno del servidor web, es decir, en el caso de sufrir un ataque exitoso y ser mitigado al instante, el servidor web sigue estando en el mismo estado que antes del ataque. Esto es un problema ya que la vulnerabilidad sigue estando presente, y el atacante puede volver a intentar explotarla.
