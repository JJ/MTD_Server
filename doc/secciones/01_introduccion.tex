\chapter{Introducción}

Los servidores son la cara visible de internet, por ejemplo, actualmente hay alrededor de 12.1 millones de servidores web activos en el mundo \cite{netcraft-agosto23}. Estos servidores son el objetivo de muchos ataques, ya que son la puerta de entrada a los datos de los usuarios.\cite{breaches-2023} Por ello, es importante que los servidores estén protegidos ante ataques, y que en caso de que se produzca un ataque, el servidor sea capaz de recuperarse y seguir ofreciendo servicio.

Hay diferentes tecnologías para aumentar la seguridad, por ejemplo los \textit{firewalls}, listas de control de acceso (ACL), \textit{firewalls} para web (WAF), \textit{Honeypots}, etc. Sin embargo, estas técnicas tienen algo en común, y es que son estáticas (se definen mediante configuraciones predefinidas). Existen otras tecnologías que pueden tomar acciones basadas en eventos como los sistemas de prevención de intrusión (IPS) o los antivirus en tiempo real. Sin embargo, ninguna de las técnicas anteriores cambia el entorno para defenderse del atacante, es decir, en el caso de sufrir una intrusión y ser mitigada, el servidor sigue estando en el mismo estado que antes del ataque. Esto es un problema, ya que la vulnerabilidad sigue estando presente, y hasta que no sea arreglada por el administrador, el atacante puede volver a intentar explotarla.

Para solucionar lo anterior, se introduce un concepto llamado \textit{moving target defense}\cite{big-state-of-art} (MTD de ahora en adelante), el cual se refiere a una estrategia de seguridad dinámica que implica cambiar la configuración o distribución de un sistema para dificultar los ataques y proteger contra las amenazas. Esta estrategia busca hacer que los sistemas sean menos predecibles para los atacantes al cambiar las condiciones bajo las cuales operan. No busca reemplazar a las técnicas existentes, sino complementarlas.

Con este trabajo se pretende facilitar el trabajo a los investigadores, implementando uno de los últimos MTD como herramienta de código abierto que les permita probar y evaluar este tipo de técnicas. Además se pretende realizar diferentes comparativas para evaluar dicha implementación, realizando así una aportación a dicha línea de investigación. Para llevar a cabo todo lo anterior, nos apoyaremos en el desarrollo ágil.

Este proyecto es software libre, y está liberado con la licencia\cite{gplv3} en https://github.com/marcosrmartin/MTD_Server.